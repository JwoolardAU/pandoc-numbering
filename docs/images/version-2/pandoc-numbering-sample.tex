\PassOptionsToPackage{unicode=true}{hyperref} % options for packages loaded elsewhere
\PassOptionsToPackage{hyphens}{url}
%
\documentclass[]{article}
\usepackage{lmodern}
\usepackage{amssymb,amsmath}
\usepackage{ifxetex,ifluatex}
\usepackage{fixltx2e} % provides \textsubscript
\ifnum 0\ifxetex 1\fi\ifluatex 1\fi=0 % if pdftex
  \usepackage[T1]{fontenc}
  \usepackage[utf8]{inputenc}
  \usepackage{textcomp} % provides euro and other symbols
\else % if luatex or xelatex
  \usepackage{unicode-math}
  \defaultfontfeatures{Ligatures=TeX,Scale=MatchLowercase}
\fi
% use upquote if available, for straight quotes in verbatim environments
\IfFileExists{upquote.sty}{\usepackage{upquote}}{}
% use microtype if available
\IfFileExists{microtype.sty}{%
\usepackage[]{microtype}
\UseMicrotypeSet[protrusion]{basicmath} % disable protrusion for tt fonts
}{}
\IfFileExists{parskip.sty}{%
\usepackage{parskip}
}{% else
\setlength{\parindent}{0pt}
\setlength{\parskip}{6pt plus 2pt minus 1pt}
}
\usepackage{hyperref}
\hypersetup{
            pdftitle={Sample use of automatic numbering},
            pdfauthor={Ch. Demko chdemko@gmail.com},
            pdfborder={0 0 0},
            breaklinks=true}
\urlstyle{same}  % don't use monospace font for urls
\setlength{\emergencystretch}{3em}  % prevent overfull lines
\providecommand{\tightlist}{%
  \setlength{\itemsep}{0pt}\setlength{\parskip}{0pt}}
\setcounter{secnumdepth}{0}
% Redefines (sub)paragraphs to behave more like sections
\ifx\paragraph\undefined\else
\let\oldparagraph\paragraph
\renewcommand{\paragraph}[1]{\oldparagraph{#1}\mbox{}}
\fi
\ifx\subparagraph\undefined\else
\let\oldsubparagraph\subparagraph
\renewcommand{\subparagraph}[1]{\oldsubparagraph{#1}\mbox{}}
\fi

% set default figure placement to htbp
\makeatletter
\def\fps@figure{htbp}
\makeatother


\title{Sample use of automatic numbering}
\author{Ch. Demko \href{mailto:chdemko@gmail.com}{\nolinkurl{chdemko@gmail.com}}}
\date{04/11/2015}

\begin{document}
\maketitle

\section*{List of theorems}
\addcontentsline{toc}{section}{List of theorems}

\hypersetup{linkcolor=black}\makeatletter\newcommand*\l@theorem{\@dottedtocline{1}{1.5em}{2.3em}}\@starttoc{theorem}\makeatother

\section*{List of examples}
\addcontentsline{toc}{section}{List of examples}

\hypersetup{linkcolor=black}\makeatletter\newcommand*\l@example{\@dottedtocline{1}{1.5em}{2.3em}}\@starttoc{example}\makeatother

\section*{List of exercises}
\addcontentsline{toc}{section}{List of exercises}

\hypersetup{linkcolor=black}\makeatletter\newcommand*\l@exercise{\@dottedtocline{1}{2.0em}{3.0em}}\@starttoc{exercise}\makeatother

\hypertarget{this-is-the-first-section}{%
\section{This is the first section}\label{this-is-the-first-section}}

\phantomsection\addcontentsline{exercise}{exercise}{\protect\numberline {1}{\ignorespaces }}\protect\hypertarget{exercise:1}{}{\label{exercise:1}\textbf{Exercise
1}}

This is the first exercise. Have also a look at the
\protect\hyperlink{theorem:first}{Theorem 1.1}, the
\protect\hyperlink{exercise:second}{exercise 2} and the exercise
\protect\hyperlink{exercise:last}{3}.

\begin{quote}
\phantomsection\addcontentsline{theorem}{theorem}{\protect\numberline {1.1}{\ignorespaces Needed for the second exercise}}\protect\hypertarget{theorem:first}{}{\label{theorem:first}\emph{Theorem
1.1: Needed for the \protect\hyperlink{exercise:second}{second
exercise}}}

This is a the first theorem. Look at the
\protect\hyperlink{exercise:second}{exercise}.
\end{quote}

\phantomsection\addcontentsline{exercise}{exercise}{\protect\numberline {2}{\ignorespaces }}\protect\hypertarget{exercise:second}{}{\label{exercise:second}\textbf{Exercise
2} \emph{(This is the second exercise)}}

Use \protect\hyperlink{theorem:first}{\emph{theorem 1.1} page
\pageref{theorem:first}}

\begin{description}
\tightlist
\item[\phantomsection\addcontentsline{example}{example}{\protect\numberline {1.1}{\ignorespaces Example}}\protect\hypertarget{example:1.1}{}{\label{example:1.1}\textbf{Example
1}}]
This is the first example of the first section
\item[\phantomsection\addcontentsline{example}{example}{\protect\numberline {1.2}{\ignorespaces Example}}\protect\hypertarget{example:1.2}{}{\label{example:1.2}\textbf{Example
2}}]
This is the second example of the first section
\end{description}

\hypertarget{this-is-the-second-section}{%
\section{This is the second section}\label{this-is-the-second-section}}

\begin{description}
\tightlist
\item[\phantomsection\addcontentsline{example}{example}{\protect\numberline {2.1}{\ignorespaces Example}}\protect\hypertarget{example:2.1}{}{\label{example:2.1}\textbf{Example
1}}]
This is the first example of the second section
\end{description}

\begin{quote}
\phantomsection\addcontentsline{theorem}{theorem}{\protect\numberline {2.1}{\ignorespaces Theorem}}\protect\hypertarget{theorem:2.1}{}{\label{theorem:2.1}\emph{Theorem
2.1}}

Another theorem. Can be useful in
\protect\hyperlink{exercise:1}{exercise 1}
\end{quote}

\begin{quote}
\phantomsection\addcontentsline{theorem}{theorem}{\protect\numberline {2.2}{\ignorespaces Theorem}}\protect\hypertarget{theorem:2.2}{}{\label{theorem:2.2}\emph{Theorem
2.2}}

A last theorem.
\end{quote}

\begin{description}
\tightlist
\item[\phantomsection\addcontentsline{example}{example}{\protect\numberline {2.2}{\ignorespaces Example}}\protect\hypertarget{example:2.2}{}{\label{example:2.2}\textbf{Example
2}}]
This is the second example of the second section
\end{description}

\phantomsection\addcontentsline{exercise}{exercise}{\protect\numberline {3}{\ignorespaces }}\protect\hypertarget{exercise:last}{}{\label{exercise:last}\textbf{Exercise
3}}

This is the third exercise.

Unnumbered \#

\end{document}
